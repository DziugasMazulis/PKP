\documentclass{VUMIFPSkursinis}
\usepackage{algorithmicx}
\usepackage{algorithm}
\usepackage{algpseudocode}
\usepackage{amsfonts}
\usepackage{amsmath}
\usepackage{bm}
\usepackage{caption}
\usepackage{color}
\usepackage{float}
\usepackage{graphicx}
\usepackage{listings}
\usepackage{float}
\usepackage{subfig}
\usepackage{wrapfig}
\usepackage[hidelinks]{hyperref}
\usepackage{todonotes}
\usepackage{xcolor}
\usepackage{lineno}
\linenumbers
% Titulinio aprašas
\university{Vilniaus universitetas}
\faculty{Matematikos ir informatikos fakultetas}
\department{Programų sistemų katedra}
\papertype{Programų kūrimo proceso laboratorinis darbas}
\title{Įmonės ,,Mėnuliukų technologijos" programų kūrimo proceso aprašas}
\titleineng{Description of the development process of the ,,Mėnuliukų technologijos" company}
\status{4 kurso 3 grupės studentai}
\author{Mėnuliukai}


\supervisor{Saulius Ragaišis, Doc., Dr.}
\date{Vilnius – \the\year}

% Nustatymai
% \setmainfont{Palemonas}   % Pakeisti teksto šriftą į Palemonas (turi būti įdiegtas sistemoje)
\bibliography{bibliografija}

\begin{document}
\maketitle

\tableofcontents
	\section{Vertinimo apimtis}
		\begin{itemize}
			\item{Vertinama pagal - CMMI-DEV, V1.3}
			\item{Vertinimo apimtis - visa antrame darbe pagerinta organizacija.}
			\item{Aukščiausias vertinamas gebėjimo lygis - maksimalus kurį gali pasiekti procesų sritis.}
			\item{Vertinamos procesų sritys:
				\begin{enumerate}
					\item{Causal Analysis and Resolution}
					\item{Configuration Management}
					\item{Integrated Project Management}
					\item{Organizational Process Definition}
					\item{Organizational Performance Management}
					\item{Organizational Process Performance}
					\item{Project Planning}
					\item{Process and Product Quality Assurance}
					\item{Quantitive Project Management}
					\item{Requirement Development}
					\item{Requirements Management}
					\item{Technical Solution}
					\item{Validation}
					\item{Verification}
				\end{enumerate}
			}
		\end{itemize}
	\section{Vertinimo rezultatai prieš pagerinimą}
	\section{Technical solution}
		\subsection{Pagerinimas}
			\subsubsection{Proceso aprašymas po pagerinimo}
		\subsection{NI PI LI FI atitikimas po pagerinimo}
	\section{Verification}
		\subsection{Pagerinimas}
			\subsubsection{Proceso aprašymas po pagerinimo}
		\subsection{NI PI LI FI atitikimas po pagerinimo}
	\section{Configuration management}
		\subsection{Pagerinimas}
			\subsubsection{Proceso aprašymas po pagerinimo}
		\subsection{NI PI LI FI atitikimas po pagerinimo}
	\section{Vertinimo rezultatai po pagerinimo}
	\section{Rezultatai ir išvados}

\end{document}
